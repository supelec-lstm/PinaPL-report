% !TeX root = main.tex

\chapter{Real time reccurent learning}

A partir de cette section, les objectifs sont de reconnaître une chaine de
caractère en temps réel : c'est à dire que en donnant un ou plusieurs caractères
, on doit être capable de prédire la fin de la chaine. Ce genre de problème peut
 être étendu à la recherche comportementale en temps réel. Pour résoudre ce
genre de problème, on utilise des réseaux neuronaux récurrents, qui ont
l'avantage de se souvenir des états précedents pour pouvoir prédire efficacement
 les états suivants; ils comportent une mémoire courte. Dans un premier temps,
nous allons nous interesser aux réseaux RTRL.

\section{Théorie}

Dans la suite, nous allons nous interesser au problème de la grammaire de Reber,
 qui servira d'échantillon de test pour RTRL.

\subsection{La grammaire de Reber}

 Une grammaire de Reber est un langage défini par l'automate déterministe
 cyclique suivant :

\begin{figure}[!ht]
\begin{center}
\includegraphics[height=5cm]{rtrl/reberGrammar.png}
\end{center}
\caption{Grammaire de Reber simple}
\end{figure}


De base, on considère une probabilité uniforme de chosir l'état suivant parmis
les états possibles suivants. La lettre $B$ et la lettre $E$ sont des lettres
indiquant simplement le début et la fin de la chaîne, elles n'ont pas d'interet
propre pour la grammaire. Les autres lettres présentes sur les arêtes peuvent
variées, mais elles doivent respecter les règles suivantes :
\medskip
\begin{itemize}
	\item Chaque lettre doit apparaitre exactement deux fois
	\item On ne peut pas obtenir deux lettres consécutives en passant par des états différents.
\end{itemize}

\vspace{\parskip}
L'interêt de la grammaire de Reber est que c'est un automate simple qui ne
nécessite que la mémoire de la dernière et de l'avant dernière lettre pour
trouver la suivante. En effet, d'après la dernière règle, connaître les deux
dernières lettres impose l'état actuel dans l'automate. En outre, chaque lettre
apparaissant deux fois, la connaissance seule de la dernière lettre ne suffit
pas à prédir la suivante correctement. On remarque que l'on peut résoudre ce
problème avec un perceptron classique si on donne en entrée du perceptron les
deux dernières lettres du mot.
