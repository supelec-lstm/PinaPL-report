% !TeX root = main.tex

\chapter{Long Short Term Memory}
Objectif principal de ce projet, l'architecture neuronale Long Short Term Memory (LSTM) est décrite dans cette partie.
Tout comme les autres architectures neuronales, elle est constituée d'un assemblage de blocs élémentaires qui disposent d'un
ensemble de variables, appelés poids, à adapter lors de la phase d'apprentissage afin de reproduire une fonction.
Cependant, la cellule élémentaire d'un réseau LSTM est bien plus complexe que celle d'un réseau neuronal à perceptrons. \\

La dénomination LSTM vient du fait que ce type de réseau possède une mémoire de plus longue durée que des structures de type RTRL.
Ainsi, il sera possible d'apprendre des fonctions telles que la grammaire de Reber double, ou bien de générer du texte après avoir appris
des écrits de Shakespeare. \\
LSTM est notamment utilisé dans des applications de reconnaissance vocale.

\section{Théorie}
\subsection{Cellule LSTM}
\subsection{Propagation}
\subsection{Algorithmes d'apprentissage}

\section{Implémentation}

\section{Résultats}
\subsection{Apprentissage sur un mot}
\subsection{Grammaire de Reber simple}
\subsection{Grammaire de Reber double}
