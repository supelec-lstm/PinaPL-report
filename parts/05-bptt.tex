% !TeX root = main.tex

\chapter{Back Propagation Through Time}
L'algorithme BPTT appliqué à un réseau neuronal récurrent a pour particularité
de déplier le temps dans l'espace; par exemple, pour apprendre un mot de cinq
caractères, on va créer cinq réseaux non-récurrents qui représentent chaqun un
"temps", soit une lettre de la séquence.

\medskip

Cet algorithme a pour avantage, par rapport à celui RTRL, d'avoir une
complexité temporelle inférieure.

\section{Théorie}

\begin{figure}[!ht]
\begin{center}
\includegraphics[width=0.8\textwidth]{images/bptt.png}
\caption{Dépliement du temps dans l'espace pour BPTT}
% TODO : crédits
\end{center}
\end{figure}


\subsection{Réseau BPTT}

\section{Implémentation}

L'implémenation est effectuée en C++ via la librairie de calcul matriciel
Eigen3. Toutes les matrices sont des objets de type Eigen::MatrixXd (matrice de
double) et les vecteurs des objets de type Eigen::VectorXd.

\medskip

L'aléatoire utilisé est celui natif en C et C++ : rand.
La génération de la graine se fait à partir du temps à la milliseconde pour
éviter une initialisation déterministe dans le cas de l'execution de plusieurs
runs consécutifs. Pour cela la librairie 'sys/time.h' est utilisée, avec un
appel propre aux systèmes UNIX.

\bigskip

\subsection{Structure de données}

Le code se décompose en plusieurs éléments : 
\begin{itemize}
  \item Les poids
  \item Une couche de neurones fully-connected
  \item Le réseau de couches dépliées dans le temps
\end{itemize}

\subsubsection{Les poids}

% TODO : détailler attributs et methodes

Enfin, les méthodes de l'objet Poids sont le constructeur et
l'application des variations de poids (qui remet par la même occasion à 0
les delta-poids).

\subsubsection{La couche de neurones}

% TODO : détailler fonction, attributs, methodes

\subsubsection{Le réseau}

% TODO : détailler fonction, attributs, methodes

\section{Résultats}
\subsection{Grammaire de Reber simple}
\subsection{Grammaire de Reber double}
