% !TeX root = main.tex

\chapter{Présentation du projet}

\section{Objectifs}

Ce projet a été lancé le 23 septembre 2016 sur proposition de Mme. Joana Tomasik
et Mr.\ Arpad Rimmel. Le but est de réimplémenter en moins d'un an l'algorithme
LSTM\footnotemark. Il s'agit d'un réseau neuronal récurrent avec mémoire.

\footnotetext{Long Short Term Memory}

\bigskip

Un tel projet représente de fortes contraintes techniques (implémentation du
 réseau, gestion de la mémoire, performances, \ldots) mais aussi un défi
 théorique (preuve de terminaison, gestion des cycles, \ldots).

\bigskip

Pour arriver à ce résultat, l'année est découpée en différents "runs" et dans
chacun nous devons découvrir et implémenter une spécificité des réseaux
neuronaux.
Tout d'abord nous voulons implémenter un réseau neuronal simple pour
nous familiariser avec le concept des réseaux de neurones et mettre en place
les outils adéquats pour les simuler. Puis, nous modifierons ces réseaux
basiques pour y introduire des propriétés plus complexes, tels les réseaux
récurrents (RTRL, BPTT). Enfin, nous adapterons le réseau et son algorithme pour
qu'il corresponde à celui du LSTM.

\section{Equipe}

L'équipe du projet est composée de 4 élèves-ingénieurs de CentraleSupélec Gif :
Maxime Amossé, Julien Hemery, Hugo Hervieux et Sylvain Pascou, encadrés par deux
enseignants-chercheurs du même établissement, Mme.\ Joana Tomasik et Mr.\ Arpad
Rimmel.

\section{Outils}

\subsection{Le langage}
Le langage de programmation choisi est le C++, pour sa rapidité d'execution,
sa plus grande versatilité que le C ainsi que le aspects pédagogiques de son
utilisation. Il est important de faire remarquer que ce langage est compilé et
donc ne permet donc pas directement à l'utilisateur d'intéragir avec les
variables durant l'exécution, ce qui complique les phases de débugage.

\subsection{Git}

Pour travailler en groupe de manière efficace nous avons utilisé le
gestionnaire de version Git et un dépot distant sur la plateforme GitHub :
\url{https://github.com/supelec-lstm/PinaPL}. Il nous a fallu une dizaine
d'heures et quelques règles de bonne conduite pour ne pas créer des conflits à
chaque merge et faciliter le travail de chacun.

\smallskip

Chaque développeur travaille donc sur la branche de la fonction qu'il
implémente puis la merge dans la branche master lorsqu'il à une version
stable qui compile et se comporte correctement.

\subsection{Zotero}

Zotero est un outil gratuit de gestion de bibliographie. Il permet à tous les
membres du projet d'y enregistrer des références utiles, des articles papiers,
mais aussi les fichiers PDF contenant lesdits articles.
De plus, il nous permet d'exporter régulièrement la bibliographie sous un
format standard pour l'inclure dans ce rapport.

\subsection{LaTeX}

Il s'agit d'un langage de création de documents. Axé autour du document
scientifique, il est reconnu par la communauté pour sa facilité d'utilisation
et sa capacité à générer des documents propres et ordonnés. Ce rapport a été
rédigé grâce à LaTeX.
