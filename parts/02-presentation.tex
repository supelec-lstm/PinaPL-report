% !TeX root = main.tex

\chapter{Présentation du projet}

\section{Objectifs}

Ce projet a été lancé le 23 septembre 2016 sur proposition de M\textsuperscript{me} Joana \textsc{Tomasik}
et M.\ Arpad \textsc{Rimmel}. Le but est d'implémenter en un peu moins d'un an un algorithme d'apprentissage sur un réseau neuronal avec mémoire appelé LSTM\footnotemark, en étudiant d'autres structures de réseaux plus simples auparavant.

\footnotetext{Long Short Term Memory}

\bigskip

Un tel projet représente de fortes contraintes techniques (implémentation de la structure de
 réseau neuronal, gestion de la mémoire, performances, \ldots) mais aussi un défi
 théorique (élaboration des algorithmes, leur justification formelle, \ldots).

\bigskip

Pour arriver à ce résultat, l'année est découpée en différents grandes étapes et dans
chacune nous devons découvrir et implémenter une spécificité des réseaux
neuronaux.
Tout d'abord nous voulons implémenter un réseau neuronal simple pour
nous familiariser avec le concept des réseaux de neurones et mettre en place
les outils adéquats pour les simuler. Puis, nous modifierons ces réseaux
basiques pour y introduire des propriétés plus complexes, tels les réseaux
récurrents (RTRL, BPTT). Enfin, nous adapterons le réseau et son algorithme pour
qu'il corresponde à celui du LSTM.

\section{Équipe}

L'équipe du projet est composée de quatre élèves-ingénieurs de CentraleSupélec, cursus ingénieur Supélec :
Maxime \textsc{Amossé}, Julien \textsc{Hemery}, Hugo \textsc{Hervieux} et Sylvain \textsc{Pascou}, encadrés par deux
enseignants-chercheurs du même établissement, M\textsuperscript{me} Joana \textsc{Tomasik} et M.\ Arpad
\textsc{Rimmel}.

\section{Outils}

\subsection{Langage de programmation}
Le langage de programmation choisi est le C++, pour sa rapidité d'exécution,
sa grande versatilité, la richesse de sa communauté, ainsi que le aspects pédagogiques de son
utilisation. Il est important de faire remarquer que ce langage est compilé et
donc ne permet donc pas directement à l'utilisateur d'interagir simplement avec les
variables durant l'exécution, ce qui complique significativement les phases de déboggage.

\subsection{Outil de versionnement}

Pour travailler en groupe de manière efficace nous avons utilisé le
gestionnaire de version Git et un dépôt partagé sur la plateforme GitHub :
\url{https://github.com/supelec-lstm/PinaPL}. Il nous a fallu une dizaine
d'heures et quelques règles de bonne conduite pour éviter de créer des
conflits et faciliter le travail de chacun.

\smallskip

Chaque développeur travaille donc sur la branche de la fonction qu'il
implémente, puis il la fusionne dans la branche principale lorsqu'il a une version
stable qui compile et se comporte correctement.

\subsection{Gestion de la bibliographie et références}

Zotero est un outil gratuit de gestion de bibliographie. Il permet à tous les
membres du projet d'y enregistrer des références utiles, des articles papiers,
mais aussi les fichiers PDF contenant lesdits articles.
De plus, il nous permet d'exporter régulièrement la bibliographie sous un
format standard pour l'inclure dans ce rapport.

\subsection{Écriture du présent rapport}

Enfin, pour rédiger ce rapport, nous avons utilisé \LaTeX\xspace. Il s'agit d'un
langage de création de documents axé autour du document
scientifique ; il est reconnu par la communauté pour sa facilité d'utilisation
et sa capacité à générer des documents propres et ordonnés.
