\documentclass[a4paper]{report}

%================================= PACKAGES ====================================

\usepackage[french]{babel}
\usepackage[utf8x]{inputenc}
\usepackage{graphicx}
\usepackage{amsmath}
\usepackage{amssymb}
% positionnement d'images
\usepackage{float}
\usepackage[colorinlistoftodos]{todonotes}
\usepackage{url}
% informations sur un document compilé en PDF et les liens externes / internes
\usepackage[]{hyperref}
% mise en page des tableaux
\usepackage{array}
\usepackage{tabularx}
% espacement entre les lignes
\usepackage{setspace}
% modifier la mise en page de l'abstract
\usepackage{abstract}
% police et mise en page (marges) du document
\usepackage[T1]{fontenc}
\usepackage[top=2cm, bottom=2cm, left=2cm, right=2cm]{geometry}
% galeries d'images
\usepackage{subfig}
\usepackage{svg}

%========================== INFORMATION ET REGLES ==============================

% numérotation pour les \paragraphe et \subparagraphe
\setcounter{secnumdepth}{4}
\setcounter{tocdepth}{4}

\setlength{\parskip}{3ex}

%\hypersetup{
%% Information sur le document
%% Auteurs
%pdfauthor = {Amosse Maxime,
%			Hemery Julien,
%			Hervieux Hugo,
%    		Pascou Sylvain},

%% Titre du document
%pdftitle = {PinaPL --- Projet long sur les réseaux neuronaux},

%% Sujet
%pdfsubject = {Compte rendu de projet},

%% Mots-clefs
%pdfkeywords = {PinaPL, neuron, \ldots},

%% Ajuste la page à la largueur de l'écran
%pdfstartview=FitH
%}

%======================== DEBUT DU DOCUMENT ========================

\begin{document}

% régler l'espacement entre les lignes
\newcommand{\HRule}{\rule{\linewidth}{0.5mm}}

% page de garde
\input{./parts/00-title.tex}

\newpage
\thispagestyle{empty}
\newpage

% résumé du projet
% !TeX root = main.tex

\renewcommand{\abstractnamefont}{\normalfont\Large\bfseries}
%\renewcommand{\abstracttextfont}{\normalfont\Huge}

\begin{abstract}
\hskip7mm

%TODO

\begin{spacing}{1.3}

Ce projet a pout but d'étudier trois grandes familles de réseaux neuronaux et d'en distinguer les capacités: les réseaux simple, les réseaux récurrents, et les réseaux \textit{LSTM} (\textit{Long Short Term Memory}). Pour chacun d'entre eux nous étudions des algorithmes d'apprentissage afin d'obtenir des résultats aussi proches que souhaités de données sources, pour des expériences variées: réalisation d'opérations logiques, reconnaissance de chiffres manuscrits, apprentissage et génération de séquences.

\end{spacing}
\end{abstract}

\pagenumbering{arabic}

% TOC
\renewcommand{\baselinestretch}{0.15}\normalsize
\tableofcontents
\renewcommand{\baselinestretch}{1.00}\normalsize

\thispagestyle{empty}
\newpage

% espacement entre les lignes d'un tableau
\renewcommand{\arraystretch}{1.5}

%====================== INCLUSION DES PARTIES ======================

\thispagestyle{empty}
% recommencer la numérotation des pages à "1"
\setcounter{page}{1}
\newpage

% parties
% !TeX root = main.tex

\chapter{Présentation du projet}

\section{Objectifs}

Ce projet a été lancé le 23 septembre 2016 sur proposition de Mme. Joana Tomasik
et Mr.\ Arpad Rimmel. Le but est de réimplémenter en moins d'un an l'algorithme
LSTM\footnotemark. Il s'agit d'un réseau neuronal récurrent avec mémoire.

\footnotetext{Long Short Term Memory}

\bigskip

Un tel projet représente de fortes contraintes techniques (implémentation du
 réseau, gestion de la mémoire, performances, \ldots) mais aussi un défi
 théorique (preuve de terminaison, gestion des cycles, \ldots).

\bigskip

Pour arriver à ce résultat, l'année est découpée en différents "runs" et dans
chacun nous devons découvrir et implémenter une spécificité des réseaux
neuronaux.
Tout d'abord nous voulons implémenter un réseau neuronal simple pour
nous familiariser avec le concept des réseaux de neurones et mettre en place
les outils adéquats pour les simuler. Puis, nous modifierons ces réseaux
basiques pour y introduire des propriétés plus complexes, tels les réseaux
récurrents (RTRL, BPTT). Enfin, nous adapterons le réseau et son algorithme pour
qu'il corresponde à celui du LSTM.

\section{Equipe}

L'équipe du projet est composée de 4 élèves-ingénieurs de CentraleSupélec Gif :
Maxime Amossé, Julien Hemery, Hugo Hervieux et Sylvain Pascou, encadrés par deux
enseignants-chercheurs du même établissement, Mme.\ Joana Tomasik et Mr.\ Arpad
Rimmel.

\section{Outils}

\subsection{Le langage}
Le langage de programmation choisi est le C++, pour sa rapidité d'execution,
sa plus grande versatilité que le C ainsi que le aspects pédagogiques de son
utilisation. Il est important de faire remarquer que ce langage est compilé et
donc ne permet donc pas directement à l'utilisateur d'intéragir avec les
variables durant l'exécution, ce qui complique les phases de débugage.

\subsection{Git}

Pour travailler en groupe de manière efficace nous avons utilisé le
gestionnaire de version Git et un dépot distant sur la plateforme GitHub :
\url{https://github.com/supelec-lstm/PinaPL}. Il nous a fallu une dizaine
d'heures et quelques règles de bonne conduite pour ne pas créer des conflits à
chaque merge et faciliter le travail de chacun.

\smallskip

Chaque développeur travaille donc sur la branche de la fonction qu'il
implémente puis la merge dans la branche master lorsqu'il à une version
stable qui compile et se comporte correctement.

\subsection{Zotero}

Zotero est un outil gratuit de gestion de bibliographie. Il permet à tous les
membres du projet d'y enregistrer des références utiles, des articles papiers,
mais aussi les fichiers PDF contenant lesdits articles.
De plus, il nous permet d'exporter régulièrement la bibliographie sous un
format standard pour l'inclure dans ce rapport.

\subsection{LaTeX}

Il s'agit d'un langage de création de documents. Axé autour du document
scientifique, il est reconnu par la communauté pour sa facilité d'utilisation
et sa capacité à générer des documents propres et ordonnés. Ce rapport a été
rédigé grâce à LaTeX.

% !TeX root = ../main.tex

\chapter{Réseau de neurones simple}

\section{Théorie}

\subsection{Le perceptron}

Le perceptron est le neurone le plus basique que l'on puisse trouver dans la
littérature. Un perceptron est défini par :

\begin{itemize}
  \item $n$ entrées $x_i$
  \item une sortie $y$
  \item $n$ poids $w_i$
  \item un biais $\theta$
  \item une fonction de composition $g : \mathbb{R}^n \to \mathbb{R}$
  \item une fonction d'activation $f : \mathbb{R} \to \mathbb{R}$
\end{itemize}

\begin{figure}[!ht]
\begin{center}
\includegraphics[width=0.75\textwidth]{images/perceptron.png}
\end{center}
\caption{Schéma d'un perceptron simple}
\end{figure}

\vspace{\parskip}
Sur sa construction, le perceptron est fortement inspiré sur le neurone humain,
sans pour autant en être une représentation réaliste.
Le perceptron détermine avec ses entrées s'il active ou non sa sortie, c'est
à dire s'il relaie le signal. Pour cela il rassemble toutes les données des
entrées $x_i$ à l'aide de la fonction de composition $g$. Le neurone ne donnant
pas la même importance à chaque entrée, on les pondère préalablement par des
poids $w_i$.
Finalement, on décide du signal de sortie à l'aide de la fonction d'activation
$f$. Le seuil d'activation est souvent centré en $0$, mais on peut le faire varier à
l'aide d'un biais $\theta$ ajouté en entrée de la fonction d'activation.

\medskip

En résumé, on a :
\[y = f(g(x_1w_1, x_2w_2, \ldots , x_nw_n) + \theta) \]

\medskip

Usuellement, la fonction de composition est une simple somme pondérée,
ce qui nous donne :
\[y = f(\sum_{i=1}^n x_iw_i + \theta) \]

\medskip

On remarque que le biais agit comme le poids d'une entrée du neurone qui serait
toujours $1$.

\subsection{Le réseau}

Une réseau de neurones permet de créer des fonctions de bien plus grande
complexité qu'un simple neurone, permettant de résoudre des problèmes
jusqu'alors inaccessibles à la machine. Il permet par exemple de résoudre le
problème de classification ; par exemple, un problème classique est celui où l'on dispose
d'images représentant des chiffres manuscrits et que l'on doit lire.
La base MNIST, que l'on utilisera par la suite, propose plusieurs dizaines de milliers
de telles images. Ce problème est d'une extrême simplicité pour un être
humain, mais presque impossible à résoudre avec une programmation classique
sans réseau neuronal.

Ainsi, pour construire un réseau, chaque perceptron est
mis en relation avec ses pairs : un perceptron prend en entrée les sorties
d'autres perceptrons. On construit alors un graphe orienté dans lequel chaque
sommet est un perceptron. On se limitera dans un premier temps au cas d'un
réseau acyclique, puis on étudiera les possibilités supplémentaires offertes par
la présence de cycles bien placés.

\bigskip

On peut définir plusieurs types de neurones dans un réseau :
\begin{itemize}
\item les neurones d'entrée
\item les neurones cachés
\item les neurones de sortie
\end{itemize}

\bigskip

\begin{figure}[!ht]
\begin{center}
  \includegraphics[width=0.3\textwidth]{images/reseau_simple.png}
\end{center}
\caption{Réseau de neurones à une couche cachée}
\end{figure}

Ces neurones vont être placés sur des couches successives. Une couche est un ensemble de neurones de même type, tel qu'ils soient reliés à aux neurones de la couche suivante, mais pas entre eux. On définit alors une couche d'entrée, une ou plusieurs couches de neurones cachés, et une couche de neurones de sortie. Ainsi, chaque couche transmet ses calculs à la suivante jusqu'à arriver à la couche de sortie.

\medskip

Tout d'abord, il faut autant de neurones d'entrée que de dimensions qu'a l'échantillon
que l'on veut soumettre au réseau. Par exemple dans le cadre du MNIST
on veut en entrée une image de dimension $28 \times 28$, on place donc $784$
neurones d'entrées. En pratique, les neurones d'entrée sont des neurones fictifs
; ils sont présents pour faciliter la construction du réseau de neurones.
En effet, ils ne sont soumis à aucun apprentissage et leur sortie est la même
que leur entrée. Nous ne les considérerons pas dans la théorie qui suit.

\medskip

Les neurones des couches cachées sont présents entre les neurones d'entrée et de
sortie. Ils sont utiles uniquement pour le calcul de résultats intermédiaires en vue de
calculer la sortie. Le nombre de couches et la taille des couches influent sur l'action
du réseau. Un réseau à multiples couches cachées sera capable de traiter des problèmes plus
complexes qu'un réseau à une seule couche cachée. Il est évident que cela augmente
néanmoins la complexité des calculs et le temps d'exécution.

\medskip

Les neurones de sortie sont ceux qui servent pour la classification de
l'échantillon d'entrée. Il doit y avoir le même
nombre de neurones de sortie que de classes différentes possibles. Ainsi dans
l'exemple du MNIST, le but est de déterminer un chiffre donné par une image :
il y a $10$ possibilités, une pour chaque chiffre. Il y a donc $10$ neurones de
sortie.

\medskip

Par la suite, on appellera $\{x_i\}_{i \leq n}$ les $n$ entrées, $\{y_i\}_{i \leq m}$
 les $m$ sorties de tous les neurones, $\{y_i\}_{m+1-M \leq i \leq m}$ les sorties des
$M$ neurones de sorties, $\{\sigma_i\}_{i \leq m}$ les fonctions d'activations et
$\{\theta_i\}_{i \leq m}$ les biais.

\medskip

On définit enfin $\{F_i\}_{i \leq m}$ tel que $j \in F_i$ si et seulement si la
sortie du neurone $j$ est reliée au neurone $i$. On peut ainsi numéroter les
poids : $\{w_{ij}\}_{i \leq m, j \in F_i} $ le poids associé à l'entrée reliant
le neurone $j$ au neurone $i$.

\medskip

D'après ce qui précède, on peut écrire pour tout $i \in [1, m]$ :

\[y_i = \sigma_i(\sum_{j \in F_i} y_jw_{ij} + \theta_i) \]

\subsection{Les réseaux en couches}

Les réseaux en couches vus ci-dessus sont très utilisés, et nous allons établir
les premiers résultats sur ces structures. On partitionne l'ensemble donc
des neurones en $K$ ensembles, chacun décrivant une couche. De plus,
chaque neurone d'une couche a comme entrées l'ensemble des sorties
des neurones de la couche précédente. On notera $\alpha^{(k)}_j$ l'élément $\alpha$
numéro $j$ de la couche $k$. On notera également $N_k$ le nombre de neurones à
la couche $k$. On a donc pour tout $k > 1$ :

\[y_i^{(k)} = \sigma_i^{(k)}(\sum_{j = 0}^{N_{k-1}} y_j^{(k-1)}w_{ij}^{(k)} + \theta_i^{(k)}) \]

On remarque que l'on peut simplifier la notation en considérant la formule
ci-dessus avec une approche vectorielle :

\[\overline{y^{(k)}} = w^{(k)} \times y^{(k-1)} + \theta^{(k)}\]
\[y^{(k)} = \sigma^{(k)}(\overline{y^{(k)}}) \]

où $\overline{y^{(k)}}$ est la somme pondérée des entrées.

On peut montrer que tout réseau acyclique peut se ramener à un réseau à couches
dont certains poids sont imposés comme nuls. Nous prenons donc l'hypothèse que
le réseau est un réseau à $K$ couches, que la première couche est l'ensemble des
 neurones d'entrée et la dernière couche l'ensemble des neurones de sortie, sans
 perte de généralité.

\subsection{L'apprentissage}

L'efficacité d'un réseau de neurones se mesure à la qualité de sa classification. 
Celle-ci dépend des poids qui sont attribués à chacune de ses entrées. Il faut
donc déterminer la bonne combinaison de poids qui permettra au réseau de
simuler la fonction voulue. Le nombre de poids présents dans un réseau augmente
très rapidement et il devient complexe d'estimer cette bonne combinaison. Pour
cela, on procède à une phase d'apprentissage : on utilise un échantillon de
données dont on connaît le résultat pour construire un réseau avec les bons poids.
On part ainsi d'un réseau avec des poids aléatoires, choisis dans un
intervalle restreint et centré en zéro, et on les modifie en prenant en compte
les erreurs entre les valeurs obtenues et les valeurs théoriques. Dans la suite,
on s'intéressera à toute la démarche nécessaire pour mettre en œuvre cette
modification de poids.

\medskip

On notera $\{x_i\}_{i \leq n}$ et $\{Y_i\}_{i \leq M}$ les entrées et sorties
des échantillons d'apprentissage.

\medskip

Pour déterminer les modifications à effectuer, on calcule la sortie du réseau de
neurones avec un échantillon de test donné et on mesure l'erreur entre cette sortie
et le résultat voulu. Pour cela,
on choisira une fonction qui mesurera la différence entre le vecteur de sortie
et le vecteur des sorties théoriques. Classiquement, on utilise la méthode des
moindres carrés $E_m$ ou bien une fonction softmax à laquelle on rajoute une
entropie croisée $E_s$:

\[E_m = \sum_{j = 1}^{M} \cfrac{(Y_j - y_j^{(K)})^2}{2}\]
\[E_s = \sum_{j = 1}^{M} Y_j \log \left(\cfrac{e^{y_j^{(K)}}}{\sum e^{y_i^{(K)}}}\right)\]

\subsection{La méthode des gradients}

On veut donc minimiser $E$ en modifiant les $w_{ij}$. Le problème ici est que
l'on a une connaissance limitée de $E$ en fonction des $w_{ij}$ car on ne
dispose des valeurs théoriques de sortie que pour un nombre fini de valeurs. Or
les méthodes de minimisation de fonctions reposent souvent sur une connaissance
continue de ce que l'on veut optimiser. La seule méthode viable est la méthode de la
descente du gradient.

\medskip

On a une fonction $f$, appelée fonction de coût, que l'on veut minimiser par
rapport à un facteur $x$. On crée alors une suite $(x_n)$ telle que
$x_{n+1} = x_{n} - \cfrac{\partial f}{\partial x}(x_n)$.
L'idée est se déplacer sur le potentiel de $f$ grâce à son gradient. Avec cette
méthode, on peut calculer facilement la suite $(x_n)$ car il suffit d'évaluer
le gradient en un point et non plus en un nombre continument infini.

\medskip

Cependant, cette méthode est imprécise et il arrive qu'elle converge vers un
minimum local. En pratique, l'ajout de neurones va augmenter le nombre de
dimensions du gradient et donc permettre de limiter le nombre de minima locaux.

\medskip

\subsection{La rétropropagation}

D'après ce qui précède, l'objectif est donc d'évaluer pour tout $w_{ij}^{(k)}$ :
 $\cfrac{\partial E}{\partial w_{ij}^{(k)}}$.

\begin{align*}
\cfrac{\partial E}{\partial w_{ij}^{(k)}} &= \sum_{l = 1}^{M} \cfrac{\partial y_l^{(K)}}{\partial w_{ij}^{(k)}} \times \cfrac{\partial E}{\partial y_l}\\
&= \left\langle\cfrac{\partial y^{(K)}}{\partial w_{ij}^{(k)}}, J_E\right\rangle\\
&= \left\langle\cfrac{\partial \overline{y^{(K)}}}{\partial w_{ij}^{(k)}} \odot \sigma'(\overline{y^{(K)}}), J_E\right\rangle \\
&= \left\langle\cfrac{\partial \overline{y^{(K)}}}{\partial w_{ij}^{(k)}} ,\sigma'(\overline{y^{(K)}}) \odot J_E\right\rangle
\end{align*}

Avec $J_E$ la jacobienne de $E$ au point considéré.
Calculons maintenant $\cfrac{\partial \overline{y^{(l)}}}{\partial w_{ij}^{(k)}}$.
Tout d'abord, remarquons que si $l = k$ alors on a simplement :

\[\cfrac{\partial \overline{y^{(k)}}}{\partial w_{ij}^{(k)}} = E_{ij} y^{(k-1)} = y^{(k-1)}_j e_i\]

Ceci vient du fait que le réseau étant acyclique, $ y^{(l)}$ ne dépend pas de
$w_{ij}^{(k)}$ pour $l < k$. De même, pour $l > k$, on obtient :

\begin{align*}
\cfrac{\partial \overline{y^{(l)}}}{\partial w_{ij}^{(k)}} &= w^{(l)} \times \cfrac{\partial y^{(l-1)}}{\partial w_{ij}^{(k)}}\\
&= w^{(l)} \times \left(\sigma(\overline{y^{(l-1)}}) \odot \cfrac{\partial \overline{y^{(l-1)}}}{\partial w_{ij}^{(k)}}\right)
\end{align*}

Donc en supposant qu'il existe un vecteur $v$ tel que $\cfrac{\partial E}{\partial w_{ij}^{(k)}} = \left\langle\cfrac{\partial \overline{y^{(l)}}}{\partial w_{ij}^{(k)}} ,v\right\rangle$ :

\begin{align*}
\cfrac{\partial E}{\partial w_{ij}^{(k)}} &= \left\langle w^{(l)} \times \left(\sigma(\overline{y^{(l-1)}}) \odot \cfrac{\partial \overline{y^{(l-1)}}}{\partial w_{ij}^{(k)}}\right) ,v\right\rangle\\
&= \left\langle\sigma(\overline{y^{(l-1)}}) \odot \cfrac{\partial \overline{y^{(l-1)}}}{\partial w_{ij}^{(k)}},{}^t \! w^{(l)} \times v\right\rangle\\
&= \left\langle\cfrac{\partial \overline{y^{(l-1)}}}{\partial w_{ij}^{(k)}},\sigma(\overline{y^{(l-1)}}) \odot ({}^t \! w^{(l)} \times v)\right\rangle\\
\end{align*}

Donc il existe $u = \sigma(\overline{y^{(l-1)}}) \odot ({}^t \! w^{(l)} \times v)$
tel que $\cfrac{\partial E}{\partial w_{ij}^{(k)}} = \left\langle\cfrac{\partial \overline{y^{(l-1)}}}{\partial w_{ij}^{(k)}} ,u\right\rangle$

Donc par récurrence, pour tout $l \geq k$ il existe $\delta y^{(l)}$ tel que :
\[\cfrac{\partial E}{\partial w_{ij}^{(k)}} = \left\langle\cfrac{\partial \overline{y^{(l)}}}{\partial w_{ij}^{(k)}} ,\delta y^{(l)}\right\rangle\]

On remarquera que $\delta y^{(l)}$ ne dépend pas de $w_{ij}^{(k)}$. On appellera
 $\delta y^{(l)}_j$ le gradient du neurone $j$ de la couche $l$. Ce gradient
 vérifie la relation de récurrence suivante d'après ce qui précède :

\[
\left \{
\begin{array}{c @{=} c}
    \delta y^{(K)} & \sigma'(\overline{y^{(K)}}) \odot J_E\\
    \delta y^{(k)} & \sigma(\overline{y^{(k)}}) \odot ({}^t \! w^{(k+1)} \times \delta y^{(k+1)})\\
\end{array}
\right.
\]

On a alors :

\begin{align*}
\cfrac{\partial E}{\partial w_{ij}^{(k)}} &= \left\langle\cfrac{\partial \overline{y^{(k)}}}{\partial w_{ij}^{(k)}} ,\delta y^{(k)}\right\rangle\\
&= \left\langle y^{(k-1)}_j e_i,\delta y^{(k)}\right\rangle\\
& = y^{(k-1)}_j \times \delta y^{(k)}_i
\end{align*}

Donc :

\[\Delta w^{(k)}_{ij} = \lambda \times y^{(k-1)}_j \times \delta y^{(k)}_i\]

\[\Delta w^{(k)} = \lambda \times (\delta y^{(k)} \times {}^t \! y^{(k-1)})\]

On obtient donc un algorithme d'apprentissage qui se fait en deux temps : tout
d'abord le calcul du gradient qui se fait récursivement, puis le calcul de la
différence de poids à appliquer. Dans ce cas, on propage le gradient de la couche
de sortie vers la couche d'entrée. C'est ceci qui donne le nom à la
méthode employée : la rétropropagation.

\section{L'implémentation}

\subsection{Le neurone}

Le neurone est une classe, il a pour attributs :
\begin{itemize}
  \item le nombre de ses entrées (\verb+int+).
  \item les valeurs de ses entrées (\verb+vector<double>+).
  \item les poids qu'il leur attribue (\verb+vector<double>+).
  \item son biais (\verb+int+).
  \item sa fonction de composition (\verb+compositionFunction+).
  \item sa fonction d'activation (\verb+activationFunction+).
\end{itemize}

\medskip

Il dispose des méthodes suivantes :
\begin{itemize}
  \item \verb+description()+ : indique l'état du neurone.
  \item \verb+reset()+ : remet à zero ses entrées et sa sortie.
  \item les getters et les setters pour les poids, le nombre d'entrées, les
  fonctions de composition...
  \item \verb+calculateOutput()+ : calcul la sortie du neurone en fonction
  de ses entrées.
  \item \verb+getActivationDerivative+ : calcul de la dérivée de la fonction
  d'activation au point observé.
  \item \verb+getCompositionDerivative+ : idem pour la fonction de composition.
\end{itemize}

\subsection{Le réseau}

Le réseau est aussi implémenté en tant que classe.
Un réseau dispose :
\begin{itemize}
  \item d'un nom (\verb+string+).
  \item d'une date de création (\verb+string+).
  \item de ses neurones (\verb+vector<Neuron>+).
  \item de la liste de ses neurones d'entrée (\verb+vector<unsigned long>+).
  \item de la liste de ses neurones de sortie (\verb+vector<unsigned long>+).
  \item de la matrice des liens entre neurones (\verb+vector<vector<double>>+).
  \item de son facteur d'apprentissage (\verb+unsigned long+).
  \item de ses valeurs en entrée (\verb+vector<double>+).
  \item de la matrice des sorties des neurones (\verb+vector<double>+).
\end{itemize}

\medskip

Il dispose des méthodes nécessaires à la propagation du signal ainsi qu'à
sa rétropropagation.

\subsection{Améliorations apportées}

Nous avons ensuite amélioré le code pour diminuer le temps de calcul et clarifier
la structure. Les éléments à améliorer sont :
\begin{itemize}
  \item La structure orientée objet
  \item Le single-threading
  \item calcul à chaque pas des sorties de chaque neurone
\end{itemize}

\medskip

Nous nous sommes progressivement débarrassés de la structure d'objet du neurone
en effectuant les conversions suivantes :

\medskip

\begin{tabular}{c|c}
   structure objet & nouvelle structure \\
   \hline
   neurones.poids + matrice des poids + matrice des relations & matrice des poids \\
   neuron.activationFunction & vecteur de fonctions d'activation \\
   neuron.compositionFunction & on ne considère plus que la somme \\
   neuron.inputs/output & vecteur des entrées/sorties de tout le réseau \\
   neuron.bias & vecteur des biais de chaque neurone du réseau \\
   dérivée de la fonction de composition & égale à $1$
\end{tabular}

\medskip

De plus, nous avons déterminé en amont les neurones voisins qui nécessitaient
un rafraîchissement de leur sortie. Cela permet de ne pas calculer à chaque
itération la sortie de tous les neurones du réseau. Lors de la création du
réseau est construite une liste de vecteurs des neurones dont il faut évaluer
la sortie au tour $i$.

\section{Résultats}

Après une longue période de déboggage, nous avons obtenu des résultats
satisfaisants.


\subsection{Le XOR}

Les premiers tests simples à réaliser avec un réseau neuronal sont les opérations logiques : en effet, de telles opérations prennent deux arguments en entrées $A$ et $B$, de valeur $0$ ou $1$, et donnent en sortie un valeur $S$ égale à $0$ ou à $1$. Cela se représente aisément avec un réseau de neurones, avec deux neurones d'entrée $A$ et $B$, et un neurone de sortie $S$ ; les neurones $A$ et $B$ sont tous deux liés deux neurones $C$ et $D$, qui forment une couche cachée, et ceux-ci sont reliés à $S$.

Si les opérations logiques rudimentaires telles que $\text{FAUX}: (A, B) \mapsto 0$, $\text{A}: (A, B) \mapsto A$ ou $\text{OR}: (A, B) \mapsto (A ∨ B)$ ne posent aucun problème, le cas de $\text{XOR}: (A, B) \mapsto (A ∧ ¬B) ∨ (¬A ∧ B)$ est plus intéressant car il requiert l'utilisation d'une couche cachée ainsi que des biais.

On réalise l'apprentissage à l'aide de 100000 entrée successives, où les valeurs de $A$ et $B$ sont choisies aléatoirement. La propagation, rétro-propagation sont effectués à chaque étape, en revanche la modification des poids est faite toute les 100 entrées. Le facteur d'apprentissage est fixé à 0.5 et les poids initialisés par des valeurs aléatoires entre $-1$ et $1$ ; on choisit la sigmoïde comme fonction d'activation pour chacun des neurones et un coût quadratique pour l'évaluation de l'erreur.

Dans ces conditions, l'apprentissage est quasiment instantané. On réalise ensuite les quatre tests correspondant aux quatre cas possibles pour les valeurs de $(A, B) \in \{0, 1\}^2$, et on regarde la valeur obtenue et le coût final $c$ défini par

\[ c = \sum_{(A, B) \in \{0, 1\}^2} (f(A, B) - (A \text{ XOR } B))^2 \]

où $f(A, B)$ est le résultat donné par le réseau de neurones.

On obtient généralement les résultats suivants, à $10^{-2}$ près:
\[\begin{split}
f(0, 0) &= 0.01 \\
f(1, 0) &= 0.99 \\
f(0, 1) &= 0.99 \\
f(1, 1) &= 0.01 \\
\text{et }c &= 0.02
\end{split}\]

On obtient donc des résultats très correct. Comme la fonction $f$ du réseau de neurone n'est pas définie exclusivement sur $\{0, 1\}^2$, mais sur $\mathbb{R}^2$, on peut visualiser l'évolution de $f(A, B)$ pour $(A, B) \in [-0.5, 1.5]^2$. Cela donne le graphique de la figure 2.2, où les valeur usuelles de $(A, B)$ sont indiquées par des points. On peut voir que le réseau à identifié une bande dans laquelle $f(A, B)$ vaut $1$, et vaut $0$ ailleurs.

\begin{figure}[!ht]
\begin{center}
  \includegraphics[width=0.5\textwidth]{images/xor1.png}
\end{center}
\caption{Résultats corrects de XOR}
\end{figure}

Cependant, dans certains cas le réseau apprend mal le XOR. Cela se traduit par la convergence vers un minimum local de la fonction de coût dans lequel le réseau va se bloquer et dont il ne pourra sortir. De tels erreurs arrivent environ dans 1 cas sur 3, et sont dues à l'initialisation aléatoire des poids. Des résultats typiques d'un mauvais apprentissage sont présentés en figure 2.3 et 2.4.

\begin{figure}[!ht]
\begin{center}
  \includegraphics[width=0.5\textwidth]{images/xor2.png}
\end{center}
\caption{Résultat erroné de XOR ($c = 0.83$)}
\end{figure}

\begin{figure}[!ht]
\begin{center}
  \includegraphics[width=0.5\textwidth]{images/xor3.png}
\end{center}
\caption{Autre résultat erroné de XOR ($c = 0.76$)}
\end{figure}

Enfin, on peut aussi introduire une erreur dont les données d'apprentissage. Cela revient à inverser la sortie souhaitée de temps en temps ; ainsi si on veut apprendre à l'étape $i$ que $0 \text{ XOR } 1$ vaut $1$, introduire une erreur reviendra à fixer à l'étape $i$ $A = 0$, $B = 1$ et $S = 0$ (au lieu de $S = 1$ donc). En faisant cela de façon régulière, on peut mesurer la résilience du réseau et sa capacité à apprendre malgré des données d'apprentissage de moindre qualité.

Par exemple, en introduisant une erreur dans $20\%$ des cas, on obtient les résultats suivants:

\[\begin{split}
f(0, 0) &= 0.11 \\
f(1, 0) &= 0.85 \\
f(0, 1) &= 0.87 \\
f(1, 1) &= 0.17 \\
\text{et }c &= 0.28
\end{split}\]

\begin{figure}[!ht]
\begin{center}
  \includegraphics[width=0.5\textwidth]{images/xor4.png}
\end{center}
\caption{Résultat avec des données erronées}
\end{figure}

Malgré la proportion d'erreurs qui peut sembler importante, on voit que le réseau arrive tout de même à séparer le bruit des bonnes valeurs. Ainsi, il est encore capable de distinguer la bande où $f$ est égale à $1$, même si la différence entre la bande et le reste est amoindrie. Cependant, le taux de mauvais apprentissages est aussi plus important, et si la proportion d'erreur est trop grande le réseau n'est plus capable d'appendre correctement.


\subsection{Le MNIST}

Les données du problème du MNIST sont réparties en deux fichiers :
\begin{itemize}
    \item Le set d'apprentissage qui contient 60000 entrées (images de $28 \times 28$ pixels
          des chiffres manuscrits suivis des données des chiffres représentés)
    \item Le set de test qui contient 10000 entrées (différentes de celle du set
          d'apprentissage)
\end{itemize}

\medskip

Les valeurs des entrées sont stockées dans les fichiers entre 0 et 255, nous
les avons centrées et normalisées (entre $-0.5$ et $0.5$).

\medskip

Tous les résultats présentés par la suite sont établis en soumettant au réseau
après apprentissage les 10000 entrées du set de test et en comparant la sortie
attendue et la sortie obtenue. On obtient le résultat du calcul du réseau en
prenant la sortie du réseau avec la valeur maximale.

\subsubsection{Le classificateur linéaire}

En créant un réseau neuronal sans biais ni couche cachées on obtient un
classificateur linéaire. Les 10 neurones de sortie sont reliés chacun aux 784
neurones d'entrées. Les poids sont initialisés aléatoirement entre -1 et 1 selon
 une loi uniforme. La fonction d'activation est une siogmoïde, la composition
est une somme pondérée, et la fonction de coût est l'écart quadratique. Le
facteur d'apprentissage $\lambda$ est fixé à $0.3$ en accord avec la littérature
 sur le sujet.

\medskip

Les résultats d'un tel réseau sont forts intéressants car permettent après un
apprentissage stochastique des 60000 échantillons de test d'obtenir un
pourcentage moyen de réussite de 88\% avec un écart-type de 0.87\%.
(Moyenne effectuée sur 100 réalisations).

\subsection{Réseau avec 1 couche cachée}

Ci-dessous la courbe d'apprentissage obtenue avec un réseau comportant une couche
cachée de 10 neurones, et avec un learning rate de 0.3.

\begin{figure}[!ht]
\begin{center}
  \includegraphics[width=0.7\textwidth]{images/mnist-256-10-03.png}
\end{center}
\caption{Courbe d'apprentissage du MNIST avec une couche cachée de 10 neurones}
\end{figure}

\subsubsection{Réseau avec deux couches cachées}

Pour un réseau comportant deux couches cachées de 300 neurones, on atteint un taux
de réussite sur 100 itérations successives de 94.66\% en moyenne avec un écart-type de 0.25\%.

% !TeX root = ../main.tex

\chapter{Real time reccurent learning}

A partir de cette section, les objectifs sont de reconnaître une chaine de caractère en temps réel : c'est à dire que en donnant un ou plusieurs caractères , on doit être capable de prédire la fin de la chaine.
Ce genre de problème peut être étendu à la recherche comportementale en temps réel.
Pour résoudre ce genre de problème, on utilise des réseaux neuronaux récurrents, qui ont l'avantage de se souvenir des états précédents pour pouvoir prédire efficacement les états suivants; ils comportent une mémoire courte.
Dans un premier temps, nous allons nous intéresser aux réseaux RTRL.

\section{Théorie}

Dans la suite, nous allons nous intéresser au problème de la grammaire de Reber, qui servira d'échantillon de test pour RTRL.

\subsection{La grammaire de Reber}

Une grammaire de Reber est un langage défini par l'automate déterministe cyclique suivant :

\begin{figure}[!ht]
\begin{center}
\includegraphics[width=0.6\textwidth]{images/reberGrammar.png}
\end{center}
\caption{Grammaire de Reber simple}
% TODO : Include credits
\end{figure}


De base, on considère une probabilité uniforme de choisir l'état suivant parmi les états possibles suivants.
La lettre $B$ et la lettre $E$ sont des lettres indiquant simplement le début et la fin de la chaîne, elles n'ont pas d'intérêt propre pour la grammaire.
Les autres lettres présentes sur les arêtes peuvent variées, mais elles doivent respecter les règles suivantes :

\medskip

\begin{itemize}
	\item Chaque lettre doit apparaitre exactement deux fois
	\item On ne peut pas obtenir deux lettres consécutives en passant par des états différents.
\end{itemize}

\medskip

L'intérêt de la grammaire de Reber est que c'est un automate simple qui ne nécessite que la mémoire de la dernière et de l'avant dernière lettre pour trouver la suivante.
En effet, d'après la dernière règle, connaitre les deux dernières lettres impose l'état actuel dans l'automate.
En outre, chaque lettre apparaissant deux fois, la connaissance seule de la dernière lettre ne suffit pas à prédire la suivante correctement.
On remarque que l'on peut résoudre ce problème avec un perceptron classique si on donne en entrée du perceptron les deux dernières lettres du mot.
Ce modèle bien que résolvant ce problème, n'est pas adapté au calcul en temps réel.
De plus il ne résout pas le problème de la grammaire de Reber double.

\medskip

Le problème de la grammaire double est un problème similaire à la grammaire
simple. L'automate le représentant est :

\begin{figure}[!ht]
\begin{center}
\includegraphics[width=0.6\textwidth]{images/reberGrammarSymmetric.png}
\end{center}
\caption{Grammaire de Reber symétrique}
\end{figure}

On remarque qu'il est constitué de deux grammaires de Reber simple qui sont reliés en entrée et en sortie.
La difficulté de ce problème est qu'il faut mémorisé la première valeur pour en déduire la dernière.
Dans ce cas, une mémoire ''infini'' est nécessaire théoriquement pour se souvenir de la première entrée.
Il est donc impensable d'utiliser de la même façon un réseau de perceptron classique. On a alors besoin de réseau récurrent, dont la sortie à un instant $t$ va dépendre de la sortie à un instant $t-1$.

\subsection{Réseau RTRL}

Le principe d'un réseau récurrent est d'utiliser le résultat obtenue par la sortie précédente à l'entrée du calcul suivant.
On aura donc en appelant $x(t)$ la donnée en entrée à l'instant $t$ et $y(t)$ la sortie associé.
On a alors $y(t) = f(x(t), y(t-1))$.
En appelant $\alpha$ un paramètre de la fonction $f$.
Le but est de faire un apprentissage sur $\alpha$ pour minimiser à chaque temps $t$ l'erreur $E(t)$ entre la sortie théorique et la sortie pratique.
De la même façon que précédemment, on va donc calculer $\cfrac{\partial E(t)}{\partial \alpha}$ pour utiliser la méthode du gradient :
Dans un premier temps, nous allons nous intéresser au cas où $f$ est de la forme :

\[ f(x(t), y(t-1)) = \sigma\left (W x(t) + R y(t-1) + b\right )\]

Avec $W$, $R$ qui sont des matrices de poids, $b$ un biais et $\sigma$ une fonction d'activation.
On reconnait donc d'après ce qui précède la formule d'un réseau perceptron à $0$ couche caché et "fully connected".
On pose $\overline{y(t)} = W\times x(t) + R\times y(t-1) + b$.
Les paramètres du systèmes sont donc les éléments de la matrice
On obtient donc pour l'apprentissage :

\[\cfrac{\partial E(t)}{\partial \alpha} = \left\langle \cfrac{\partial y(t)}{\partial \alpha}, J_E \right\rangle\]

\begin{figure}[!ht]
\begin{center}
  \includegraphics[width=0.5\textwidth]{images/rtrl.png}
\end{center}
\caption{Réseau récurrent RTRL}
\end{figure}

\section{Implémentation}

\section{Résultats}
\subsection{Grammaire de Reber simple}
\subsection{Grammaire de Reber double}

% !TeX root = main.tex

\chapter{Back Propagation Through Time}

\section{Théorie}
(dépliement du temps dans l'espace,\ldots)

\subsection{Réseau BPTT}

\section{Implémentation}

\section{Résultats}
\subsection{Grammaire de Reber simple}
\subsection{Grammaire de Reber double}

% !TeX root = main.tex

\chapter{Long Short Term Memory}
Objectif principal de ce projet, l'architecture neuronale Long Short Term Memory (LSTM) est décrite dans cette partie.
Tout comme les autres architectures neuronales, elle est constituée d'un assemblage de blocs élémentaires qui disposent d'un
ensemble de variables, appelés poids, à adapter lors de la phase d'apprentissage afin de reproduire une fonction.
Cependant, la cellule élémentaire d'un réseau LSTM est bien plus complexe que celle d'un réseau neuronal à perceptrons. \\

La dénomination LSTM vient du fait que ce type de réseau possède une mémoire de plus longue durée que des structures de type RTRL.
Ainsi, il sera possible d'apprendre des fonctions telles que la grammaire de Reber double, ou bien de générer du texte après avoir appris
des écrits de Shakespeare. \\
LSTM est notamment utilisé dans des applications de reconnaissance vocale.

\section{Théorie}
\subsection{Cellule LSTM}
\subsection{Propagation}
\subsection{Algorithmes d'apprentissage}

\section{Implémentation}

\section{Résultats}
\subsection{Apprentissage sur un mot}
\subsection{Grammaire de Reber simple}
\subsection{Grammaire de Reber double}

%% !TeX root = main.tex

%Ne pas numéroter cette partie
\part*{Annexes}
%Rajouter la ligne "Annexes" dans le sommaire
\addcontentsline{toc}{part}{Annexes}

%\input{./annexes/annexe1.tex}
%\input{./annexes/annexe2.tex}


\newpage

% récupérer les citation avec "/footnotemark"
\nocite{*}
% inclusion de la biblio
\bibliography{./bibliographie.bib}
% choix du style de la biblio
\bibliographystyle{plain}
% voir wiki pour plus d'information sur la syntaxe des entrées d'une
% bibliographie

\end{document}
